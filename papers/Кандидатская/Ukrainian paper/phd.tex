%%
%% This is file `xampl-thesis.tex',
%% generated with the docstrip utility.
%%
%% The original source files were:
%%
%% vakthesis.dtx  (with options: `xampl-thesis')
%% 
%% IMPORTANT NOTICE:
%% 
%% For the copyright see the source file.
%% 
%% Any modified versions of this file must be renamed
%% with new filenames distinct from xampl-thesis.tex.
%% 
%% For distribution of the original source see the terms
%% for copying and modification in the file vakthesis.dtx.
%% 
%% This generated file may be distributed as long as the
%% original source files, as listed above, are part of the
%% same distribution. (The sources need not necessarily be
%% in the same archive or directory.)
% Існують кілька опцій, які необхідно вказувати як факультативний
% аргумент команди \documentclass. Наприклад, для докторської
% дисертації необхідно написати
% \documentclass[d]{vakthesis}

% Налагодження кодування шрифта, кодування вхідного файла
% та вибір необхідних мов

\documentclass[14pt]{vakthesis}
\usepackage{extsizes}
\usepackage{cmap} % для кодировки шрифтов в pdf
\usepackage[T1,T2A]{fontenc}
\usepackage[utf8]{inputenc}
\usepackage[english,ukrainian]{babel}

%\usepackage{libertine}
%\usepackage[letterspace=-36]{microtype}

% Підключення необхідних пакетів. Наприклад,
% Пакети AMS для підтримки математики, теорем, спеціальних шрифтів
% \usepackage{phd-thesis-library}
\usepackage{packages}
\allowdisplaybreaks


% Налагодження параметрів сторінки (зокрема берегів).
% Наприклад, за допомогою пакета geometry
\usepackage{geometry}
\geometry{hmargin={30mm,15mm},lines=29,vcentering}

% Якщо потрібно працювати лише з деякими розділами
%\includeonly{xampl-ch1,xampl-bib}

% Інформація про використані пакети тощо.
% Може знадобитися для відлагодження класу документа
%\listfiles



% Титульна сторінка
\makeatletter
\def\onesupervisorname{Науковий\
    керівник}%
  \def\manysupervisorsname{Наукові\
    керівники}
\makeatother 

\makeatletter
%\def\@maketitle{%
%  {\scshape
%   \@ifundefined{@institution@office}{\relax}{\@institution@office\par}
%   \@institution\par}
%  \vspace{\stretch{3}}%
%  {\raggedleft \@ifundefined{@secret}{}{\@secret\hfill}%
%     На правах\
%     рукопису \par}%
%  \vspace{\stretch{2}}%
%  {\bfseries\expandafter\emphsurname\@author \par}%
%  \vspace{\stretch{2}}%
%  {\raggedleft \CYRU\CYRD\CYRK\ \@udc \par}%
%  \vspace{\stretch{1}}%
%  {\large\bfseries\scshape \@title \par}%
%  \vspace{\stretch{2}}%
%  \@specialitycode\ --- \@specialityname\par
%  \vspace{\stretch{2}}%
%  Дисертація\
%    на\ здобуття\
%    вченої\
%    ступені\linebreak[1]%
%    \degreename\cyra\
%    \@science\par%?\linebreak[1]%
%  \vspace{\stretch{2}}%
%  {\raggedleft
%   \let\\\@format@person
%   \ifx\@supervisor\@empty
%     \hbox{}\hbox{}\hbox{}
%   \else
%     \@supervisors@caption\@supervisors\relax
%   \fi
%   \par}%
%  \vspace{\stretch{3}}%
%  \@town\ --- \@year}
%\def\@supervisors@caption{%
%  \ifnum\value{@supervisors@count}>1
%    \manysupervisorsname:%
%  \else
%    \onesupervisorname
%  \fi}
%\def\@format@person#1#2{\linebreak[4]%
%  \textbf{#1}, \linebreak[1]\@@format@person#2,,\@nil
%  \futurelet\next\@delimit@person}
%\def\@@format@person#1,#2,#3\@nil{#1%
%  \if\relax#2\relax\else, \linebreak[0]#2\fi}
%\def\@delimit@person{\ifx\relax\next\else,\fi}
%\def\emphsurname#1 #2{\textsc{#1} #2}
%\makeatother 

% Локальні означення
%\hyphenpenalty=10000
%New commands
%
\newcommand{\bs}[1]{\ensuremath{\boldsymbol{#1}}}
\newcommand{\len}[1]{|#1|}
\newcommand{\pto}{\dashrightarrow}
\newcommand{\diver}{\!\uparrow}
\newcommand{\conver}{\!\downarrow}
\newcommand{\conveq}{\mathrel{\downarrow\!=}}
\newcommand{\var}[1]{\mbox{\texttt{#1}}}
%
\newcommand{\down}{\conver}
\newcommand{\downeq}{\conveq}
\newcommand{\pmaps}{\pto}
\newcommand{\conv}{\conver}


%=======================================
%

% Environments
\theoremstyle{plain}
\newtheorem{problem}{������}[chapter]
\newtheorem{example}{������}[chapter]
\newtheorem{proposition}{�����������}[chapter]
\newtheorem{corollary}{���������}[chapter]
\newtheorem{lemma}{�����}[chapter]
\newtheorem{theorem}{�������}[chapter]

\theoremstyle{definition}
\newtheorem{definition}{�����������}[chapter]

\theoremstyle{remark}
\newtheorem{remark}{���������}[chapter]
%================================================
%%% ��������������� ���������� %%%
\renewcommand{\abstractname}{���������}
\renewcommand{\alsoname}{��. �����}
\renewcommand{\appendixname}{����������}
\renewcommand{\bibname}{������ �������������� ����������}
\renewcommand{\ccname}{���.}
%\renewcommand{\chaptername}{������}
\renewcommand{\chaptername}{������}
\renewcommand{\contentsname}{����������}
\renewcommand{\enclname}{���.}
\renewcommand{\figurename}{���.}
\renewcommand{\headtoname}{��.}
\renewcommand{\indexname}{���������� ���������}
\renewcommand{\listfigurename}{������ ��������}
\renewcommand{\listtablename}{������ ������}
\renewcommand{\pagename}{���.}
\renewcommand{\partname}{�����}
\renewcommand{\refname}{������ �������������� ����������}
\renewcommand{\seename}{��.}
\renewcommand{\tablename}{����.}
\renewcommand{\proofname}{��������������}
\renewcommand{\algorithmcfname}{��������}% Update algorithm name
%
%% ��������� ������������ � ����������
%\renewcommand{\cftpartleader}{\cftdotfill{\cftdotsep}} % for parts
%\renewcommand{\cftchapleader}{~~\cftdotfill{\cftdotsep}} % for chapters
%\renewcommand{\cftsecleader}{\cftdotfill{\cftdotsep}} % for sections, if you really want! (It is default in report and book class (So you may not need it).


  
\begin{document}

%\setdefaultleftmargin{0pt}{}{}{}{}{}

% Назва дисертації
\title{Математичні імітаційні моделі для забезпечення узгодженості для розподілених сховищ даних}
% Прізвище, ім'я, по батькові здобувача
\author{Жолткевич Галина Григоріївна}
% Прізвище, ім'я, по батькові наукового керівника/консультанта
\supervisor{Рукас Кирило Маркович}
% Науковий ступінь, вчене звання наукового керівника/консультанта
           {доктор технічних наук, доцент}
% Спеціальність
%\speciality{01.05.02}
% Варіант із вказуванням факультативних аргументів
\speciality[Математичне моделювання та обчислювальні методи]{01.05.02}[технічних наук]
% Індекс за УДК
\udc{004.042/519.713.2}
% Установа, де виконана робота, і місто
\institution{Міністерство освіти і науки України \linebreak Харківський національний університет імені~В.Н.~Каразіна}{Харків}
	% Рік, коли написана дисертація
\date{2019}

% Тут буде титульна сторінка
\maketitle

% Зміст
\tableofcontents

\chapter*{Вступ}

\paragraph{\bfseries Актуальність теми.}

В наше сьогодення інноваційні технології з'являються дуже швидко, а існуючі розвиваються з неймовірною швидкістю. Гіперлуп, багаточисленні дослідження космосу, наукові роботи в інших галузях, таких, 
як медицина, зелені мережі, а також, більш побутові, але все ще такі потрібні технології, такі, як 
комунікації, транспорт, розумні будинки... Не можна нехтувати тим фактом, що всі ці системи потребують
більшої гнучкості, швидкості, надійності та засобів для зберігання інформації також надійно та швидко і доступно, а інколи навіть доступність має бути майже у будь-якій точці земної кулі.
Тому одним з найважливіших компонентів для багатьох таких систем є швидке і надійне розподілене сховище.
В 21 столітті термін "розподілене сховище" становиться вже звичним. Деякі сховища збільшують кількість вузлів, деякі - ні. Причиною цьому є те, що багато з таких систем потребують сильної узгодженості даних. Але якщо збільшувати кількість вузлів для сховища, консистентність падає дуже швидко. А для деяких систем це важлива частина для їх стабільної роботи.

Бо є дуже відома CAP-теорема, яка стверджує, що неможливо одночасно задовільнити всі три 
характеристики для сховища, узгодженість (consistency), доступність (availability), 
стійкість до розділення (partition tolerance).

Ми не збираємося оскаржувати цю теорему, але робимо спробу обійти цю проблему. Механізм для цього і буде темою для цієї роботи.

\paragraph{\bfseries Взаємозв'язок роботи з науковими програмами, планами, темами.}
Диссертаційна робота виконана згідно з планом \linebreak
научно-исследовательских работ Харьковского национального университета имени В.Н. Каразина 
в рамках темы "Математическое и компьютерное моделирование информационных процессов в сложных естественных и технических системах" 
(номер государственной регистрации 0112U002098).


\paragraph{\bfseries Мета і завдання дослідження.}
Метою роботи являється побудува імітаційних та математичних моделей для механізму
підтримки сильної узгодженості у розподілених сховищах даних, проведення експериментів, оцінювання складності імітаційних моделей, побудува метрик, за якими можна дослідити складність даних моделей, а також розробка
обчислювальних методів для сформованого механізму. Це дозволить оцінити, наскільки можна розширити будь-яку розподілену систему і сформує методи для коректної роботи за такими умовами.

Для доягнення цієї мети у роботі розв'язані наступні задачі:
%\begin{enumerate}[widest=9999,itemindent=*,leftmargin=0pt]
%\item 
%Проведен анализ проявившихся в теории и практике программной инженерии тенденций, 
%состоящих в изменении концепций разработки архитектурных решений больших систем обработки информации. 
%Это позволило сделать вывод о расширении использования асинхронных архитектурных решений основанных на событийном управлении, 
%что обосновывает актуальность задачи совершенствования математических моделей, 
%используемых для статического анализа, спецификации и проектирования систем этого класса.
%
%\item 
%Проведен анализ существующих событийно-управляемых архитектурных стилей с целью выявления их инвариантных свойств, 
%которые должны отражаться в общей системной модели событийно-управляемой системы обработки информации. 
%По результатам проведенного анализа построена общая системная модель компонента событийно-управляемой системы обработки информации. 
%
%\item 
%Исследованы свойства потоков сложных событий на предмет возможности распознавания в них заданных множеств сложных событий.
%
%\item 
%Обосновано применение абстрактных предавтоматов для математического моделирования компонентов событийно-управляемых систем обработки информации 
%путем доказательства двойственности абстрактных предавтоматов и CEP-машин.
%
%\item 
%Рассмотрен вопрос алгоритмической реализуемости CEP-машин. 
%Выявлены возможные аномалии вычислимости таких машин и предложен метод их устранения.
%
%\item 
%Разработан специальный класс CEP-машин распознающих сложные события описываемые регулярными языками. 
%А также, на базе технологий машинного обучения, построен метод их синтеза
%
%\item 
%Разработаны прототипы программных утилит статического анализа, 
%обеспечивающих верификацию программных компонентов событийно-управляемых систем обработки информации, 
%которые реализуют предложенные и обоснованные в работе вычислительные методы.
%
%\item 
%Проведена проверка прототипов программных утилит статического анализа путем их использования при разработке программного обеспечения.
%\end{enumerate}

\chapter{Балансування узгодженості}

Наскільки нам відомо, що відповідно до CAP-теореми можна задовільнити тільки будь-які дві з трьох характеристик
для розподіленого сховища даних.
В цьому розділі розглядається можливість досягнути компромісу та забезпечити консистентні відповіді 
від бази даних, не втрачая доступності бази.

Ми розуміємо, що консистентність

 In this chapter we make attempt to reach compromise and ensure or increase consistency value for any distributed datastore without losing availability and partition tolerance.
We would like to introduce the idea first and then estimate its impact availability and partition tolerance.

So let us go deeper into a problem.

Imagine we have a distributed datastore that has N nodes. We do not take into account for now role of each node
(master or slave) and assume that each node can accept read and write requests. Now we focus on the mechanism
of database request accepting and processing. If one of nodes accepted write request, we claim that the
datastore system will reach fast enough of consistency value to maintain consistent response. 

What if to try to get round this problem by handling request to database by the list of consistent nodes for. appropriate to the dataunit incoming in the request.
We want to estimate how it may impact availability and partition tolerance if we want to 
get round consistency issue this way.

There are several solutions. But they all have general architecture:
(Picture of architecture)

\end{document}