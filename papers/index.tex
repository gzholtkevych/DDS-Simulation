%\emph{\emph{•}}
\documentclass{llncs}
%
\usepackage{amsmath}
%\usepackage{amsfonts}
\usepackage{amssymb}
\usepackage{graphicx}
\usepackage{url}
\newcommand{\keyterms}[1]{\par{\bfseries{ Key Terms: }}#1}

\begin{document}
\title{Distributed Datastores: Consistency Metric for Distributed Datastores}
\author{Kyrylo Rukkas\and Galyna Zholtkevych}
\institute{V.N.~Karazin Kharkiv National University\\
	4, Svobody Sqr., 61022, Kharkiv, Ukraine\\
	\email{galynazholtkevych1991@gmail.com}}
\maketitle
\begin{abstract}
Distributed databases being evolved require write requests to be handled independently to increase availability.
This means that nodes in a database can process modifying operation. This involves huge conflicts and leads to
full inconsistency and mess between replicas. This paper investigates a trade-off between availability and consistency
in a distributed database.
\end{abstract}

\section{Introduction}\label{sec:intro}
In evolving distributed systems it is hard and important to keep replicas up-to-date.

\section{Evolving consistency metric}
The consistency question poses the following questions:
\begin{itemize}
\item Conflicts between replicas. This problem is raised and partially solved in 
\cite{bib:c_ts}
\item Consistency convergence. This depends on many factors including network topology,
network failures, links bandwidth, bottlenecks in network overall etc.
\item Broadcast storm problem
\item Loops when broadcasting data.
\end{itemize}

\section{What impacts on consistency}\label{sec:consistency}
How fast consistency will cover all the graph with one message.
\section{How to avoid broadcast problem}
\section{How to avoid conflicts with lot of concurrent write requests}
\section{Experimental part}\label{sec:experiments}
Investigation what is better for fast consistency convergence:
group of masters close enough each to other (this almost replaces centralized server)
or group of masters allocated in far each from other but having close at least one of slaves.

This needs statistic experiment

\section{Conclusion}

\begin{thebibliography}{99}

\bibitem{bib:c_ts}
Sanjay Kumar Madria: 
Handling of Mutual Conflicts in Distributed Databases using Timestamps,
The Computer Journal. Vol. 41, No.\,6 (1998) 

\end{thebibliography}

\end{document}

