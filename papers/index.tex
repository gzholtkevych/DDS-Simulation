%\emph{\emph{•}}
\documentclass{llncs}
%
\usepackage{amsmath}
%\usepackage{amsfonts}
\usepackage{amssymb}
\usepackage{graphicx}
\usepackage{url}
\usepackage[usenames]{color}

\newcommand{\keyterms}[1]{\par{\bfseries{ Key Terms: }}#1}

\begin{document}
\title{Distributed Datastores: Consistency Metric for Distributed Datastores}
\author{Kyrylo Rukkas\and Galyna Zholtkevych}
\institute{V.N.~Karazin Kharkiv National University\\
	4, Svobody Sqr., 61022, Kharkiv, Ukraine\\
	\email{galynazholtkevych1991@gmail.com}}
\maketitle
\begin{abstract}
Distributed databases being evolved require write requests to be handled independently to increase availability.
This means that nodes in a database can process modifying operation. This involves huge conflicts and leads to
full inconsistency and mess between replicas. This paper investigates a trade-off between availability and consistency
in a distributed database.
\end{abstract}

\section{Introduction}\label{sec:intro}
Supporting replicas of an evolving distributed system up-to-dates is very important but hard problem.
Given that the ACID strategy can not be supported for systems of this class \cite[{\color{red} references to the CAP-conjecture})]{?}, we need to provide a set of indicators of main characteristics of the distributed data store to assess the risk of a wrong decision because of the data inconsistency or unavailability.
In this paper, we focus our study on the problem of estimating data consistency in a distributed data store.

\section{Evolving Consistency Metric}
{\color{red}
Perhaps it is suitable to append this section to the previous.%
}\par\noindent
Given that in a distributed data store consistency of replicas may only be eventually \cite[{\color{red} you need to refer to the corresponding papers}]{?}, we need the answering the following questions
\begin{enumerate}
\item
How to resolve conflicts between replicas?
This problem is raised and partially solved in \cite{bib:c_ts}, where algorithm with timestamps on replicas is proposed.
But merging updated replicas, it takes the response time, so consistency convergence decreases.
In the paper further sections we will investigate faster consistency convergence.
\item
Is it always possible to ensure the convergence to the consistency of data and by what means can this be achieved?
This depends on many factors including network topology,
network failures, links bandwidth, bottlenecks in network overall etc \cite[{\color{red} links are needed}]{?}.
\item
What methods can be used to effectively solve the broadcast storm problem?
This is well-known problem and it has the solution in \cite[{\color{red} links are needed}]{?}.
So we will not focus on this issue.
\item How do loops when broadcasting data be eliminated?
This is what networking algorithms aimed to solve \cite[{\color{red} links are needed}]{?}.
\end{enumerate}

From this list we can focus on the second question, which cover also replicas conflicts question.

% metrics theory
How fast consistency will cover all the graph with one message.

{\color{red}%
Maybe you meant:
How many time is needed to provide consistency in the whole network in response on one write transaction?%
}
\section{One Metric to Assess Consistency of Data}
\section{Simulation Model to Assess Inconsistency Ratio of Data}
\section{Case Study}\label{sec:experiments}
Based on proposed hypothesis we want to check ... .
Investigation what is better for fast consistency convergence:
group of masters close enough each to other (this almost replaces centralized server)
or group of masters allocated in far each from other but having close at least one of slaves.

This needs statistic experiment

\section{Conclusion}

\begin{thebibliography}{99}

\bibitem{bib:c_ts}
Sanjay Kumar Madria: 
Handling of Mutual Conflicts in Distributed Databases using Timestamps,
The Computer Journal. Vol. 41, No.\,6 (1998) 

\end{thebibliography}

\end{document}

